\chapter{Introduction}
% \label{sec:chapter1}
The increasing of global temperature and the worsening of the air quality are posing a real problem for the environment. The changes observed in Earth’s climate are primarily driven by human activities, particularly fossil fuel burning. The main problem with fossil fuels is that during the process of combustion in addition to produce energy, used for everyday purposes, also \ch{CO_2} (carbon dioxide) is emitted.\\
In order to improve the situation, the 2015 Paris Agreement set an ambition to limit global warming to well below $\SI{2}{\degreeCelsius}$ above pre-industrial levels and pursue efforts to limit it to $\SI{1.5}{\degreeCelsius}$ - in part by pursuing net carbon neutrality by 2050. The substantial reduction of global greenhouse gas emissions (including \ch{CO_2})  will limit the increase of global temperature \cite{french_conference}. Countries were asked to go through a process of decarbonization: the reduction of carbon dioxide emissions through the use of low carbon power sources. These low carbon power sources usually are renewable energies such as sun, wind, geothermal heat and other natural sources. \\
Thanks to this emerging trend of decarbonization, more and more renewable power energy devices are introduced in the distribution networks. These devices bring in some complications for the distribution of power and voltage in the networks, indeed the grids are moving from unidirectional power flow (from the distribution system to the consumers) to a bidirectional power flow (in this case the consumers are also producers and the exceed energy can be transported from the consumers to the distribution system). This switch from unidirectional to bidirectional power flow requires a smarter system that can handle in an efficient way the production and distribution of voltage. In the literature, this smarter way to control a distribution system is known as active network management (\gls{ANM}) and it refers to the design of control schemes that modulate the generators, the loads, and the distributed energy storage (\gls{DES}) connected to the grid. \\
%https://www.researchgate.net/figure/A-bidirectional-system-with-distributed-generation_fig2_286569839
This voltage control problem has been studied for years, but it only comes under the spotlight in recent years for the increasing number of distributed resources introduced in the networks. It is important to control the voltage in an electrical power system for a regular operation of the electrical equipment, to prevent damage such as overheating of generators and motors, to reduce transmission losses and to maintain the ability of the system to last and prevent voltage collapse.
%https://electrical-engineering-portal.com/how-reactive-power-is-helpful-to-maintain-a-system-healthy
In particular, it is useful and needed to reduce the output of renewable generators from what they could otherwise have produced given the available resources, often referred to as the process of curtailment. Such generation curtailment, along with storage and transmission losses, constitute the principal sources of energy loss that could be minimised with \gls{ANM} \cite{gym-anm}. \\

\noindent Controlling the voltage in an active way has many interesting properties:
\begin{itemize}
    \item It is a combination of local and global problem: the voltage at each node is influenced by the powers of all other nodes, but the impact depends on the distance between them.
    \item It is a constrained optimization problem where the constraint is to keep the voltage in a given range and the objective is to minimise the total power loss.
    \item Voltage control has a relatively large tolerance, and there are no severe consequences if the control fails to meet the requirements for short periods of time. \cite{wang2022multiagent}
    \item It is a hierarchical problem where there is much information at the top of the pyramid (distribution stations and substations) and they decrease on the base of the pyramid (houses, factories) mainly due to the absence of many sensors.
\end{itemize}


