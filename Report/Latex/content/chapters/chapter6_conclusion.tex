\chapter{Conclusions and further works}
\label{ch6}
This master's thesis has investigated the analysis and the use of machine learning techniques to forecast some possible over voltage problems in a power network and to carry out some possible active control of its devices to reduce the number of critical situations.\\

In particular, for the forecasting part, different combinations of information were tested, depending on what kind of data a distribution system operator can have access to. These tests were performed using three values of $h$: 2, 4, and 16 corresponding to 30 minutes, 1 hour and 4 hours of the past network information to consider. Moreover, different artificial neural networks were trained and tested using some evaluation metric to understand their performances.\\
The results showed that it is possible to forecast the network's critical situation, in particular the most performing model is the recurrent neural network with 30 minutes of past buses' voltage information, obtaining an accuracy of 0.981 and a F1-score of 0.826.\\
Some techniques for unbalanced databased were applied as well, improving the score of the least performing model and keeping them unchanged for the best performing one.\\

For the controlling part, a reinforcement learning training method was used to train an agent to avoid over voltages problems. The model used was a deep deterministic policy gradient algorithm. This algorithm was able, using only  15 minutes of the network's past information, to solve the critical situations in the network. In particular, the agent was able to solve 100\% of the over voltage problems and, most of the time, 100\% of the under voltage problems as well; with a cost for the distribution system operator of few MWs of active power in the entire time span considered of 7 months and 13 days.\\

Some future works may examine \emph{a)} predict more time steps in the future, \emph{b)} merge the tasks, forecasting and controlling parts, with a single model, for example under the reinforcement learning framework that could predict and control at the same time; \emph{c)} improve the agent's performance to get even lower values of active and reactive power control; \emph{d)} apply the same procedure to a larger network to check whether the pipeline is robust; \emph{e)} study cases where not all the information about the devices are available due to missing data, lack of sensors and so on.
