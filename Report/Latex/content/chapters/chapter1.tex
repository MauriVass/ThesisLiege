\chapter{Introduction}
% \label{sec:chapter1}
The increasing of global temperature and the worsening of the air quality are posing a real problem for the environment. The changes observed in Earth’s climate are primarily driven by human activities, particularly fossil fuel burning. These fossil fuels are used to produce the energy, consumed for everyday purposes, but during the process they emit \ch{CO_2} (carbon dioxide).\\
In order to improve the situation, the 2015 Paris Agreement set an ambition to limit global warming to well below $\SI{2}{\degreeCelsius}$ above pre-industrial levels and pursue efforts to limit it to $\SI{1.5}{\degreeCelsius}$ - in part by pursuing net carbon neutrality by 2050. The substantial reduction of global greenhouse gas emissions (including \ch{CO_2})  will limit the increase of global temperature \cite{french_conference}. Countries were asked to go through a process of decarbonization: the reduction of carbon dioxide emissions through the use of low carbon power sources. These low carbon power sources usually are renewable energy such as sun, wind, geothermal heat. \\
Thanks to this emerging trend of decarbonization, more and more renewable power energy devices are introduced inside the distribution networks. These devices bring in some complications for the distribution of power and voltage in the networks, indeed the grids move from unidirectional power flow (from the distribution system to the consumers) to a bidirectional power flow (in this case the consumers are also producers and the exceed energy can be transported from the consumers to the distribution system). This switch from unidirectional to bidirectional power flow requires a smarter system that can handle in an efficient way the production and distribution of voltage. In the literature, this smarter system is known as active network management (\gls{ANM}) and it refers to the design of control schemes that modulate the generators, the loads, and the distributed energy storage (\gls{DES}) connected to the grid. \\
%https://www.researchgate.net/figure/A-bidirectional-system-with-distributed-generation_fig2_286569839
This voltage control problem has been studied for years, but it only comes under the spotlight in recent years for the increasing number of distributed resources introduced in the networks. It is important to control the voltage in an electrical power system for a regular operation of the electrical power equipment, to prevent damage such as overheating of generators and motors, to reduce transmission losses and to maintain the ability of the system to last and prevent voltage collapse.
%https://electrical-engineering-portal.com/how-reactive-power-is-helpful-to-maintain-a-system-healthy
In particular, it is useful and needed to reduce the output of renewable generators from what they could otherwise have produced given the available resources, often referred to as the process of curtailment. Such generation curtailment, along with storage and transmission losses, constitute the principal sources of energy loss that could be minimized with \gls{ANM} \cite{gym-anm}. \\

Active voltage control (local and global problem, constrained optimization problem)

Optimal power flow

Low observability of the network/grid
% use [] to set name for ToC
% \section{Problem} % ok with fontsize=12pt

