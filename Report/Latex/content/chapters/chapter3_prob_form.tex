\chapter{Problem formulation}

A distribution network can be represented as a direct graph $\mathcal{G}(\mathcal{N},\mathcal{E})$, where \gls{N} is a set of positive integers representing the buses (or nodes) in the network, \gls{E} $\subseteq \mathcal{N} \times \mathcal{N}$ is the set of directed edges linking two buses together. The notation \gls{e} $\in \mathcal{E}$ refers to the directed edge with sending bus $i$ and receiving bus $j$. Each bus might be connected to several devices, which may inject or withdraw power from the grid. The set of all devices is denoted by \gls{D} that can be either loads \gls{L} or generators \gls{G}. \\
Several variables are associated with each bus $i \in \mathcal{N}$: a bus voltage $\mathcal{V}_i$, a bus current injection $I_i$, an active power injection $P_i$ and reactive power injection $Q_i$. The complex powers $S_i$, $S_d$ $\in \mathbb{C}$ injected into the network at bus $i$, or device $d$, can be obtained by the relation $S_i = P_i + \mathbf{i}Q_i$ or $S_d = P_d + \mathbf{i}Q_d$. \\
Similarly, variables $I_{ij}$, $P_{ij}$, $Q_{ij}$ and $S_{ij}$ refer to the direct flow of the quantities in branch $e_{ij}$ as measured at bus $i$ \cite{gym-anm}.

\section{Problem statement}
\label{sec:ProbStat}
With the increasing number of renewable energy generators entering the power distribution systems, more and more stress is placed on the networks. This energy produced by the generators can increase the voltage in certain points of the network, causing blackouts and even worse damaging the network's transmission lines. These are interruptions of the energy distribution and they must be avoided. \\

A possible solution to this problem is to control the renewable energy generators to produce less than what they could have produced; known as curtailment process. \\
In a distribution network, there are some information for the elements connected to the grid, like active and reactive power, magnitude voltage value and so on. These values are recorded and expressed as a multi-dimentional time series.\\

More formally, given the directed graph \gls{G}, at timestamp $t$ it is possible to know the feature $X=[l_p,l_q,b_V,e_L]$ with $l_p$ and $l_q$ are the active and reactive power of each load $l \in \mathcal{L}$, $b_V$ are the buses voltage and $e_L$ are the loadings of the edge connecting 2 buses. It is possible that $X$ is known, or only a subset of it, $X_s \subset X$ is known due to non-fully observability of the network. This $X$ time series dataset will be the input of an artificial neural network \gsl{ANN}.\\
The problem can be solved as a classification or regression problem: 
\begin{itemize}
    \item as a classification problem, the output of the \gsl{ANN} will be a binary value that states if in the next $n$ time steps there will be an over or under voltage problem. For the loss, many loss functions can be used, for example a binary cross-entropy loss function. Given $p(y_i) \in \{0,1\}$ the output probability prediction given the input $x \in X$ (or $x_c \in X_c$) the loss would be calculated as follows: 
    \[
    H = - \frac{1}{N} \sum_{i=1}^{N} y_i \log{(p(y_i))} + (1 - y_i) \log{(1 - p(y_i))} 
    \]
    where $y_i$ is the real label, $p(y_i)$ is the predicted probability that there will be a voltage problem and N is the number of inputs.
    \item for the regression problem, the output of the \gsl{ANN} will be the voltage forecast for each buses for the next $n$ time steps. Also for this problem, many loss functions can be used, for example a \gls{MAE} loss function, calculated as follows:
    \[
    MAE = \frac{1}{T \cdot N} \sum_{t=1}^{T} \sum_{i=1}^{N} |y_{t,i} - \hat{y}_{t,i}|
    \]
    where $y_{t,i}$ is the real voltage value at time $t$, $\hat{y}_{t,i}$ is the predicted voltage at time $t$ and $T$ is the number of future time steps to forecast.
\end{itemize}

The distribution network used is the MV Oberrhein network from Pandapower and the time series are taken from the Simbench dataset. This database refers to some real distribution networks in Germany in the year 2016; the dataset spans over one year with a $\Delta t$ of 15 minutes (a total of 35,040 time steps).

\section{Optimization problem}
This voltage control problem can be formulated as a stochastic problem, where the goal is to minimise the losses while meeting some constraints. In particular, the objective is to minimise the loss $L_g$ of each generator $g \in$ \gls{G} due to the curtailment and avoid not useful transport of energy and other minor energy losses $L_{\epsilon}$ (read: L of epsilon). Moreover, the system has to satisfy the condition of loads demands, that the network is considered safe and the congestion risk of each edge $r_{e_{ij}}$ is less than a given maximum tolerated congestion risk $R_{e_{ij}}$, e.g. {1\%} and that the power generated by each generator $\bar{g}_p$ is less or equal to its maximum installed capacity $g^{max}_p$, and the active and reactive power of the loads ($\bar{l}_p$, $\bar{l}_q$) are in their limits. \\
Mathematically (\cite{haulogypaper}):
\[
min \sum_{g \in G} L_g + L_{\epsilon}
\]
subject to
\begin{equation*}
\begin{aligned}
& r_{e_{ij}} < R_{e_{ij}} \qquad & \forall e \in \mathcal{E} \\
& g^{min}_p \leq \bar{g}_p \leq g^{max}_p \qquad & \forall g \in \mathcal{G} \\
& l^{min}_p \leq \bar{l}_p \leq l^{max}_p \qquad & \forall l \in \mathcal{L} \\
& l^{min}_q \leq \bar{l}_q \leq l^{max}_q \qquad & \forall l \in \mathcal{L} \\
\end{aligned}
\end{equation*}

