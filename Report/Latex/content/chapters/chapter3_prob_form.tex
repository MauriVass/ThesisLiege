\chapter{Problem analysis}
\label{chapter4}
%phd report
%https://roadnighttaylor.co.uk/news/what-is-active-network-management/
The principle of active network management \gls{ANM} is to address congestion and voltage issues via short-term decision making policies \cite{ANMQuentin}. \\
\gls{ANM} creates a smarter network infrastructure providing automated control of various components in the network and provides the information needed to ensure that every device performs in an optimal manner. This automated control allows grid companies to avoid reinforcing the network with expensive upgrades, so reducing the costs.
For example, in case of energy generation from the renewable devices higher than what a particular line can handle, a grid company, to avoid congestions and possible overvoltages, has three main options:
\begin{itemize}
    \item Replace the existing line with a line that can handle a higher voltage.
    \item Add another parallel line.
    \item Use \gls{ANM}.
\end{itemize}
The first two solution require some infrastructure investment that can be expensive and troublesome, especially in the case of overhead or underground lines.\\
The solution with \gls{ANM} does not require construction cost for the grid company; to keep the network working, in this case, the output of the renewable devices can be curtailed to reduce lines' overloading. \\

In these references, \gls{ANM} schemes maintain the system within operational limits by relying on the curtailment of the generator devices, \glspl{PV}, \glspl{WP} and other \gls{DER} devices. \\
Curtailment of renewable energy may be seen as counter-intuitive on the environmental point of view, and it may be considered as last option. Indeed, this process can slow down the switch to clean energy, because of the lost of the curtailed energy. \\

In this mindset, \gls{ANM} could also be used to control flexible loads and reduce the curtailment effects. These flexible loads, also known as virtual batteries, such as water heaters, air conditioning systems, electric vehicles, can be controlled to be turned on if the energy production is higher than the energy consumption so to avoid curtailment on the generators \cite{flexibleloads}. \\
Another way to reduce the energy curtailment is to use Flexible Alternating Current Transmission System (\gls{FACTS}) devices. They offer some
level of power flow control and enhance the transfer capability over the existing network. This flexibility can be utilized for congestion
mitigation and renewable energy integration. Particularly, \gls{FACTS} devices allow controlling all parameters that determine active and reactive power transmission: nodal voltages magnitudes and angles and line reactance. These devices replace the mechanical switches with semiconductor switches allowed much faster response times. One problem with these devices is the cost a system operator should sustain to implement them in the network \cite{facts}.


% \section{Network topology}
% \label{sec:nt}
% A distribution network can be represented as a direct graph $\mathcal{G}(\mathcal{N},\mathcal{E})$, where \gls{N} is a set of positive integers representing the buses (or nodes) in the network, \gls{E} $\subseteq \mathcal{N} \times \mathcal{N}$ is the set of directed edges linking two buses together. The notation \gls{e} $\in \mathcal{E}$ refers to the directed edge with sending bus $i$ and receiving bus $j$. Each bus might be connected to several devices, which may inject or withdraw power from the grid. The set of all devices is denoted by \gls{D} that can be either loads \gls{L} or generators \gls{G}. \\
% Several variables are associated with each bus $i \in \mathcal{N}$: a bus voltage $\mathcal{V}_i$, a bus current injection $I_i$, an active power injection $P_i$ and reactive power injection $Q_i$. The complex powers $S_i$, $S_d$ $\in \mathbb{C}$ injected into the network at bus $i$, or device $d$, can be obtained by the relation $S_i = P_i + \mathbf{i}Q_i$ or $S_d = P_d + \mathbf{i}Q_d$. \\
% Similarly, variables $I_{ij}$, $P_{ij}$, $Q_{ij}$ and $S_{ij}$ refer to the direct flow of the quantities in branch $e_{ij}$ as measured at bus $i$ \cite{gym-anm}.\\
% \emph{I think this part should be somewhere but I am not sure where to put it. Here it looks like it is not properly linked to the previous or following part.}


\section{Problem statement}
\label{sec:ps}
This thesis will focus on \gsl{ANM} and the problem faced by a \gsl{DSO} to maintain the network within its operational limits. In particular, the system operator will evaluate whether in a given moment there will be a voltage problem, so that it would be possible to proceed with some actions, like applying curtailment to generator devices, to maintain the voltage inside a safe range.\\

The \gls{DSO} knows some information about the network:
\begin{itemize}
    \item The network topology: the number of buses, loads and generators, the lines' length, the distance between the connected buses, and the distance between each load and generator from the external grid. Moreover, the impedance of the lines is known.
    
    \item The active and reactive power of some loads at each time step. Some values may not available mainly for two reasons: \emph{a)} for that particular load there are no measurements at all due to privacy reasons or for the lack of sensors or \emph{b)} communication problem so the sensor recording was lost.
    
    \item The type of \gsl{DER} device connected to each bus. If the \gls{DER} device is directly connected to the medium voltage grid, the active power and reactive power are considered known. 
\end{itemize}

This data is used to calculate the power flow of the network and obtain more information like the voltage magnitude at each bus, lines' loading and other values. This calculation can be performed with any power system analysis tool, for this thesis Pandapower will be used.\\

%Moreover, since the power flow depends on the power injection of the different elements, it is possible to create some possible cases changing the active and reactive power of the devices. These cases can be generated multiplying the time series by some scaling factors to increase or decrease the injection values.  \\

%https://arxiv.org/ftp/arxiv/papers/2102/2102.05657.pdf
The \gls{DSO} will consider the behaviour of the network over a set of discrete time steps $t \in \{1,2,...,T-1,T\}$ of length $\Delta t$, with, $T \in \mathbb{N}$, the last time step of the time series’ horizon. A time step is considered as a snapshot of the system at one particular point in time. \\ 

The \gls{DSO} can have access to some information, let's define the information domain as $\textbf{I}$. This information can be divided in static information like the network topology, and the observation $O_t$ collected at every time step $t$ like the loads and generators' active and reactive power and the buses' voltage magnitude, lines' loading and other information. The information at time step $I_t$ can be defined as:
\begin{align*}
    I_t & \in \textbf{I}, \text{with} \, t \in \{1,2,...,T-1,T\} \\
    I_t & = (\text{static information}, O_1, O_2, ...)
\end{align*}

The operator will predict if the system, in some given $t+n$ future time steps, will be in a critical condition. The network is considered in a critical condition if any of its elements is in an unsafe situation, for example an overvoltage (or under voltage) condition. Let define $\textbf{C}$ as the list of critical situation and $C_t$ the critical situation at time step $t$.  \\
For predicting whether the system will be in a critical situation or not, the \gls{DSO} will consider the history of the system only for $h$ preceding steps, with $h \in \mathbb{N}^+$. \\

Let also define the function $f_\mathcal{N}$ that given the information from a particular network can elaborate this information and output some values, that represent the forecasting of a critical situation. \\
It is possible to summarize all the information under the relationship:
\begin{equation} \label{eq:fmapping}
    %C = f_\mathcal{N}(I_{t-h+1},I_{t-h+2},\dots,I_{t-1},I_{t})
    \hat{\textbf{C}} = f_\mathcal{N}(\textbf{I})
\end{equation}
\noindent where $\hat{\textbf{C}}$ represents the forecasted values of the system given the information $\textbf{I}$. A single instance $\hat{C}_{t+i}$ ($\hat{C}_{t+i} \in \{0,1\}$) states if the system is critical ($\hat{C}_{t+i}=1$) or not ($\hat{C}_{t+i}=0$), with $i \in \{1,2,\dots,n-1,n\}$ and $n \in \mathbb{N}^+$. \\

% \begin{itemize}
    % \item $\textbf{I}$ is the information domain about the network. %and $I_i$ is the information vector, with $I \in \textbf{I}$, of the system at one generic time step $i$.
    % $\textbf{I}$ represent some information like the network topology, the loads and generators' active and reactive power and the buses' voltage magnitude, lines' loading and other information.
    
    % \item $h$ is the number of how many history time steps will be considered. $h \in \mathbb{N}^+$.
    
    % \item $f_\mathcal{N}$ is a generic function that maps the input information of the system to the critical system evaluation, and the subscript $\mathcal{N}$ refers to the fact that the function depends on the particular network considered. \\
    % Henceforth, the function will be referenced without the $\mathcal{N}$ subscript.% $P: \mathbb{R}^{h,n} \mapsto \{0,1\}^f$.
    
%     \item $\hat{\textbf{C}}$ represents the forecasted values of the system given the information $\textbf{I}$. A single instance $\hat{C}_{t+i}$ ($\hat{C}_{t+i} \in \{0,1\}$) states if the system is critical ($\hat{C}_{t+i}=1$) or not ($\hat{C}_{t+i}=0$), with $i \in \{1,2,\dots,n-1,n\}$ and $n \in \mathbb{N}^+$. 
%     %The network is considered in a critical situation when the following condition holds
% \end{itemize}

The main objective is to find a good mapping function $f_\mathcal{N}$ such that the actual critical values $\textbf{C}$ and forecasted values $\hat{\textbf{C}}$ are as close as possible.

\section{Solving methodology}
\label{sec:sm}
The goal of this thesis is to find a practical function $f_\mathcal{N}$. This function, that can be considered as a pipeline, mainly consist of some steps:
\begin{itemize}
    \item Solve the partial-observability problem, generating realistic time series or filling the missing measurements for the different elements. These time series are needed to run the power flow calculation. \emph{Remove?}
    \item Process the information to build a dataset, consisting of $\{\textbf{x,y}\}$ couples.
    \item Use the generated dataset to train a classifier that can forecast whether the network will be in a critical situation.
\end{itemize}

\subsection{Partial-observability problem}
%https://arxiv.org/pdf/2108.08350.pdf
As mentioned, the network has only a partial-observability mainly due to \emph{a)} values not available at all or \emph{b)} few missing values. \\

Missing values problems can be solved filling these measurements with some techniques: balancing the flow of the network or using generative models in the case of \emph{a)} or using the values from the adjacent time steps, averages or more complex methods like for example interpolation in the case of \emph{b)}. \\

These methods will allow complementing the network time series and having sufficient data to perform the power flow calculation.

\subsection{Processing}
Given all the information about the network, only a subset will be used to train the classifier, in particular the information given from the power flow calculation, that can be the voltage magnitude at each bus or the lines' loading. \\

This subset of information is divided in windows of length $h$ for the history time series, that correspond to the inputs $\textbf{x}$s and in outputs $\textbf{y}$s, of length $n$ of future time steps, that correspond to the labels.

% \begin{equation*}
%     \textbf{x} & \subseteq \textbf{I}
% \end{equation*}

% \begin{equation*}
%     \begin{aligned}
%         \textbf{y} & = C \\
%     & = [C_{t+1},C_{t+2}, \dots, C_{t+f-1},C_{t+f}]
%     \end{aligned}
% \end{equation*}
%In particular, the \gls{ANN} inputs consist of state variables from time instance $t-h+1$ to $t$. According to \ref{eq:fmapping} and \ref{eq:systemState}, the input can be expressed as a matrix of state variable:

\subsection{Forecasting}
It is common in time series forecasting problems to use artificial neural networks (\glspl{ANN}) to find a solution, thanks to their capacity to learn an approximate mapping function from the input space to the output space. In this case, the \gls{ANN} will take as input the information from the network, and it will output binary values, stating if there will be or not a critical situation in the network. \\

\noindent Given the input $\textbf{x}$, the \gls{ANN} must output some binary values stating whether the system is safe or in a critical situation at the time steps $t+n$. \\

This database, given by the couple of elements \{\textbf{x,y}\}, will be used with supervised learning techniques that may extract a forecasting function in order to solve the problem. \\

%The dataset will be split in train, validation and test set in order to train and validate the \gls{ANN}.

% \section{Problem statement}
% \label{sec:ProbStat}
% [\emph{old}] With the increasing number of renewable energy generators entering the power distribution systems, more and more stress is placed on the networks. This energy produced by the generators can increase the voltage in certain points of the network, causing blackouts or damaging the network's transmission lines. These are interruptions of the energy distribution, and they must be avoided. \\

% A possible solution to this problem is to control the renewable energy generators to produce less than what they could have produced; known as curtailment process. \\
% In a distribution network, there are some information for the elements connected to the grid, like active and reactive power, magnitude voltage value and so on. These values are recorded and expressed as a multi-dimentional time series.\\

% More formally, given the directed graph \gls{G}, at timestamp $t$ it is possible to know the feature $X=[l_p,l_q,b_V,e_L]$ with $l_p$ and $l_q$ are the active and reactive power of each load $l \in \mathcal{L}$, $b_V$ are the buses voltage and $e_L$ are the loadings of the edge connecting 2 buses. It is possible that $X$ is known, or only a subset of it, $X_s \subset X$ is known due to non-fully observability of the network. This $X$ time series dataset will be the input of an artificial neural network \gsl{ANN}.\\
% The problem can be solved as a classification or regression problem: 
% \begin{itemize}
%     \item as a classification problem, the output of the \gsl{ANN} will be a binary value that states if in the next $n$ time steps there will be an over or under voltage problem. For the loss, many loss functions can be used, for example a binary cross-entropy loss function. Given $p(y_i) \in \{0,1\}$ the output probability prediction given the input $x \in X$ (or $x_c \in X_c$) the loss would be calculated as follows: 
%     \[
%     H = - \frac{1}{N} \sum_{i=1}^{N} y_i \log{(p(y_i))} + (1 - y_i) \log{(1 - p(y_i))} 
%     \]
%     where $y_i$ is the real label, $p(y_i)$ is the predicted probability that there will be a voltage problem and N is the number of inputs.
%     \item for the regression problem, the output of the \gls{ANN} will be the voltage forecast for each buses for the next $n$ time steps. Also for this problem, many loss functions can be used, for example a \gls{MAE} loss function, calculated as follows:
%     \[
%     MAE = \frac{1}{T \cdot N} \sum_{t=1}^{T} \sum_{i=1}^{N} |y_{t,i} - \hat{y}_{t,i}|
%     \]
%     where $y_{t,i}$ is the real voltage value at time $t$, $\hat{y}_{t,i}$ is the predicted voltage at time $t$ and $T$ is the number of future time steps to forecast.
% \end{itemize}

% The distribution network used is the MV Oberrhein network from Pandapower and the time series are taken from the Simbench dataset. This database refers to some real distribution networks in Germany in the year 2016; the dataset spans over one year with a $\Delta t$ of 15 minutes (a total of 35,040 time steps).

% \section{Optimization problem}
% This voltage control problem can be formulated as a stochastic problem, where the goal is to minimise the losses while meeting some constraints. In particular, the objective is to minimise the loss $L_g$ of each generator $g \in$ \gls{G} due to the curtailment and avoid not useful transport of energy and other minor energy losses $L_{\epsilon}$ (read: L of epsilon). Moreover, the system has to satisfy the condition of loads demands, that the network is considered safe and the congestion risk of each edge $r_{e_{ij}}$ is less than a given maximum tolerated congestion risk $R_{e_{ij}}$, e.g. {1\%} and that the power generated by each generator $\bar{g}_p$ is less or equal to its maximum installed capacity $g^{max}_p$, and the active and reactive power of the loads ($\bar{l}_p$, $\bar{l}_q$) are in their limits. \\
% Mathematically (\cite{haulogypaper}):
% \[
% min \sum_{g \in G} L_g + L_{\epsilon}
% \]
% subject to
% \begin{equation*}
% \begin{aligned}
% & r_{e_{ij}} < R_{e_{ij}} \qquad & \forall e \in \mathcal{E} \\
% & g^{min}_p \leq \bar{g}_p \leq g^{max}_p \qquad & \forall g \in \mathcal{G} \\
% & l^{min}_p \leq \bar{l}_p \leq l^{max}_p \qquad & \forall l \in \mathcal{L} \\
% & l^{min}_q \leq \bar{l}_q \leq l^{max}_q \qquad & \forall l \in \mathcal{L} \\
% \end{aligned}
% \end{equation*}

