\begin{abstract}
The recent aspirations for a more sustainable energy system and the reduction of \gls{CarbonDiox}, have started a transformation in the power networks. Traditionally considered as a passive system, the power grid is undergoing a rapid change with the introduction of new distributed energy resources () and new loads (). Their introduction requires a better control of the network to ensure reliability and avoid losses. A technical consequence of these devices is the increased number of network's problems, like over voltages of the lines. The network infrastructure can reach its operational limits because of the reverse high power flow; especially when the generation is greater than the consumption.  \\
This thesis intends to investigate the possible solution when dealing with these grid changes. It also suggests different techniques to address the current and future challenges of network operation and planning. The main focus will be on medium-voltage network and one-year time series analysis of the measures of its devices. Furthermore, some machine learning models will be implemented and compared to forecast some possible over voltage problems in the network system.
\end{abstract}

\renewcommand{\abstractname}{Abstract (italian version)}
\begin{abstract}
\end{abstract}