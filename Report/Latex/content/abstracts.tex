\paginavuota

\begin{abstract}
The recent aspirations for a more sustainable energy system and the reduction of \gls{CarbonDiox} have started a transformation in the power networks. Traditionally considered as passive systems, the power grids are undergoing a rapid change with the introduction of more distributed energy resources. Their introduction requires a better control of the networks to ensure reliability and avoid energy losses. A technical consequence of these devices is the increased number of network's problems, like over voltages of the lines. These problems could damage the devices connected to the grid, with social and economic consequences. \\%The network infrastructure can reach its operational limits because of the reverse high power flow; especially when the generation is greater than the consumption.  \\
This thesis intends to investigate some possible solutions when dealing with the issues introduced by these grid changes. It also suggests different techniques to address the challenges of network forecast and control. In particular, to test whether it is possible to predict and respond in time to solve the voltage problems in the network system, some machine learning models are implemented to forecast and to control the network's devices. Two main learning algorithms are used: supervised learning, for the forecasting part; and reinforcement learning, for the controlling part.\\
The thesis focuses on a medium-voltage network and the analysis of one-year time series measurements of its devices. The time series are built using the SimBench dataset, and they are adapted to the MV Oberrhein network in order to have a real network with realistic time series.\\
The methods' results revealed that, starting from the network's devices measurements, it is possible to forecast the over voltage problems with a certain level of accuracy and in a similar way control these devices to reduce the number of voltage issues.
\end{abstract}

\paginavuota

\renewcommand{\abstractname}{Riassunto}
\begin{abstract}
Le recenti aspettative di un sistema energetico più sostenibile e la riduzione delle emissioni di anidride carbonica hanno dato il via a una trasformazione delle reti elettriche. Tradizionalmente considerate come sistemi passivi, le reti elettriche stanno subendo un rapido cambiamento con l'introduzione di un numero sempre maggiore di risorse energetiche rinnovabili. La loro introduzione richiede un migliore controllo delle reti per garantire la sicurezza ed evitare sprechi di energia. Una conseguenza tecnica di questi dispositivi è l'aumento di problemi sulla rete elettrica, come le sovratensioni delle linee elettriche. Questi problemi potrebbero danneggiare i dispositivi collegati alla rete, con conseguenze sociali ed economiche. \\
Questa tesi intende studiare alcune possibili soluzioni nell'affrontare i problemi introdotti da questi cambiamenti all'interno della rete. Suggerisce inoltre diverse tecniche nell'affrontare le sfide della previsione e del controllo del sistema elettrico. In particolare, per verificare se sia possibile prevedere e rispondere in tempo per risolvere i problemi di tensione nella rete, sono stati implementati alcuni modelli di apprendimento automatico. Vengono utilizzati due principali algoritmi di apprendimento: l'apprendimento supervisionato, per la parte di previsione, e l'apprendimento rinforzato, per la parte di controllo.\\
La tesi si concentra su una rete di media tensione e sull'analisi delle serie temporali di un anno dei suoi dispositivi. Le serie temporali sono state costruite utilizzando il dataset SimBench e sono state adattate alla rete MV Oberrhein, in modo da avere una rete reale con serie temporali realistiche.\\
I risultati dei due metodi hanno rivelato che, a partire dalle misure dei dispositivi della rete, è possibile prevedere i problemi di sovratensione con un certo livello di precisione e, allo stesso modo, controllare i dispositivi per ridurre il numero di problemi di tensione.
\end{abstract}